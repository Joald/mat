\documentclass{article}
\usepackage[english,polish]{babel}
\usepackage{polski}
\usepackage[utf8]{inputenc}
\usepackage{amsfonts}
\usepackage{soul}
\usepackage{textcomp}
\usepackage{color}
\usepackage{enumitem}
\usepackage{amsmath}
%\usepackage{amssymb}
\usepackage{mathtools}
\usepackage{listings}
\usepackage{amsthm}
\usepackage[a4paper, total={6in, 10in}]{geometry}
\definecolor{dkgreen}{rgb}{0,0.6,0}
\definecolor{gray}{rgb}{0.5,0.5,0.5}
\definecolor{mauve}{rgb}{0.58,0,0.82}
\lstset{frame=tb,
    language=Bash,
    aboveskip=3mm,
    belowskip=3mm,
    showstringspaces=false,
    columns=flexible,
    basicstyle={\small\ttfamily},
    numbers=none,
    numberstyle=\tiny\color{gray},
    keywordstyle=\color{blue},
    commentstyle=\color{dkgreen},
    stringstyle=\color{mauve},
    breaklines=true,
    breakatwhitespace=true,
    escapeinside={(*}{*)},          % if you want to add LaTeX within your code
    tabsize=4
}
\theoremstyle{remark}
\newtheorem*{lemma}{Lemma}

\title{Warte umienia, ciekawe zagadnienia}
\author{}
\date{}
\begin{document}
\maketitle
\section{Wzory skróconego mnożenia}
\begin{align}
  (a+b)^2&=a^2+2ab+b^2\\
  (a-b)^2&=a^2-2ab+b^2\\
  a^2-b^2&=(a+b)(a-b)
\end{align}
\section{Układy równań liniowych}
\subsection{Terminologia i notacja}
Po to, by było do bólu jasne, o czym jest mowa później.
(Potencjalnie) nowe pojęcia wprowadzam pogrubione.
\begin{itemize}
  \item \textbf{Równanie liniowe} nad zmiennymi $x, y$
  to każde równanie, które można sprowadzić do postaci
  $ax + by = c$, gdzie $a, b, c$ są znane, np. $3x+2y=-8$.
  Jest to \textbf{postać standardowa}.
  Analogicznie definiujemy równanie liniowe
  nad większymi ilościami zmiennych
  (ważne jest, by każda zmienna była pomnożona tylko przez
   stałą, tj. coś co nie jest zmienną).
  \subitem Liczby $a, b, c$ nazywam \textbf{współczynnikami} -
  \textbf{współczynnik} to po prostu liczba która jest przez coś pomnożona
  (zwykle przez coś najważniejszego, czyli w równaniu jest to zmienna).
  Czasami to słowo używam konkretnie do liczby, która jest pomnożona przez zmienną,
  ale dopiero po sprowadzeniu do postaci standardowej.
  Na przykład w równaniu $7x = \tfrac{3x}2 + 2$, współczynnik otrzymujemy dopiero po
   uproszczeniu:
  $\frac{11}2x = 2$ (tu oczywiście nastąpiło odjęcie $\frac32x$ od obu stron).
  W ten sposób zmienna $x$ ma tylko jeden współczynnik.
  \item \textbf{Zbiór} to kolekcja elementów (no shit, Sherlock),
    która nie zawiera dwóch tych samych elementów.
  \begin{itemize}
    \item Zbiór pusty oznaczamy przez $\emptyset$.
    \item Zbiór zawierający pojedynczy element (w tym przypadku liczbę 1)
  oznaczamy przez $\{1\}$ (więcej elementów można dodać po przecinku, np. $\{1, 2, 3\}$
  ale nie będzie to tutaj potrzebne). Jednoelementowy zbiór nazywa się \textbf{singleton}.
    \item Zbiór wszystkich liczb rzeczywistych oznaczamy przez $\mathbb R$
  (odręcznie pisane jako wielka litera R z podwójną nóżką).
  \end{itemize}
  \item \textbf{Układ równań liniowych} to ciąg równań, które jednocześnie uznajemy za prawdziwe.
  Przykłady:
  \begin{align*}
    &\left\{
    \begin{array}{lllll}
      3x &+& 2y &=& 5\\
      4x &+& 7y &=& 3
    \end{array}
    \right. \\
    &\left\{
    \begin{array}{lllllll}
      \tfrac12x &+& 2y &-&z &=& 5\\
       && 23y &-&77z &=& 100\\
      4x &+& \tfrac74y &+&9z &=& 3
    \end{array}
    \right.
  \end{align*}
  Oczywiście, układ równań jest zdefiniowany nad tymi zmiennymi, nad
  którymi są zdefiniowane jego równania.
  Ale, \textit{możemy} też sobie powiedzieć,
  że pewien konkretny układ jest nad innymi zmiennymi.
  Wtedy powiemy sobie, że każde z jego równań również
  jest nad tymi zmiennymi.
  \item Do każdego układu równań liniowych określamy \textbf{zbiór rozwiązań
  dla zmiennej} $x$.
  Chodzi o wszystkie możliwe wartości, jakie ta zmienna może przyjmować
  pod warunkiem, że wszystkie równania są prawdziwe.
  Są trzy przypadki:
  \begin{center}
    \begin{tabular}{|c|c|}
      \hline
      możliwe wartości $x$ &zbiór rozwiązań dla $x$\\
      \hline\hline
      tylko jedna wartość (oznaczę ją jako $w$) & $\{w\}$\\
      \hline
      $x$ jest dowolne & $\mathbb R$\\
      \hline
      żadna wartość nie jest możliwa & $\emptyset$\\
      \hline
    \end{tabular}
  \end{center}

  \item \textbf{Rozwiązanie} układu równań liniowych to
  przypisanie zmiennych, nad którym ten układ jest zdefiniowany do
  zbiorów rozwiązań dla tych zmiennych. Banalny przykład:
  Rozwiązaniem układu równań $
    \left\{
      \begin{array}{l}
        x=3\\
        y=2
      \end{array}
    \right.$ jest przypisanie $x \to \{3\}, y \to \{2\}$.
  \textit{Można zauważyć, że takie przypisanie też jest funkcją.}

  \item Układ równań określamy jako \textbf{oznaczony}, jeśli jego
  rozwiązanie przypisuje do \textit{każdej} zmiennej singleton.
  Np. rozwiązanie z poprzedniego podpunktu jest dowodem na to, że ów układ
  jest oznaczony. Mówimy, że taki układ \textbf{ma jedno rozwiązanie}.
  \item Układ równań określamy jako \textbf{sprzeczny}, jeśli jego
  rozwiązanie przypisuje do \textit{przynajmniej jednej} zmiennej zbiór pusty ($\emptyset$).
  Mówimy, że taki układ \textbf{nie ma rozwiązań}.
  \item W przeciwnym wypadku, tj. jeśli rozwiązanie przypisuje do
  \textit{przynajmniej jednej} ze zmiennych $\mathbb R$,
  a do innych (jeśli istnieją) singleton,
  układ jest \textbf{nieoznaczony}.

\end{itemize}
\subsection{Spostrzeżenia przydatne do rozwiązywania zadań}
Jak już zauważyliśmy IRL, układy równań z dwiema zmiennymi można rozwiązywać w
następujący sposób:
\begin{enumerate}
  \item Przekształć jedno z równań tak, by po jednej stronie była
  jedna z zmiennych, np. $x$. W ten sposób uzyskujemy
  wzór na $x$ zależny tylko od $y$.
  \item Wstaw tak wyznaczony $x$ do drugiego równania,
  w ten sposób uzyskując równanie bez $x$.
  \item Z tego równania można łatwo wyznaczyć wartość $y$.
  \item Znając $y$, wstawiamy tą wartość do wzoru wyznaczonego w punkcie 1
  i otrzymujemy wartość $x$.
\end{enumerate}
Jak nietrudno zauważyć, ten sposób działania zadziała również przy większej
liczbie zmiennych niż 2. Jednak szybko się okazuje, że takie liczenie jest powolne
i nudne, komu by się chciało w ten sposób rozwiązywać. Spójrzmy zatem na krok 1, czy nie
dałoby się go przyspieszyć. Zwróćmy uwagę, że przekształceń na równaniu dokonujemy
poprzez wykonywanie działań po obu stronach równania. Np.
dodając/odejmując coś do/od obu stron lub mnożąc/dzieląc obie strony przez jakąś liczbę
(oczywiście pamiętając, żeby nie dzielić przez 0!).
Zauważmy jednak, że przekształcając jedno z równań, mamy pewną wiedzę,
którą możemy wykorzystać - resztę równań! W końcu one też są prawdziwe, więc możemy z
jednej strony naszego równania dodać
(odjąć, nawet pomnożyć i podzielić, choć to jest radziej przydatne)
jedną stronę innego równania, a z drugiej drugą.
To się nazywa \textbf{dodawanie równań stronami} (oraz odejmowanie itd.).
Ale po co nam to? Przydatność tego sposobu łatwo zilustrować takim przykładem:
\begin{displaymath}
  \left\{
    \begin{array}{lllll}
      3x &-& 2y &=& 5\\
      4x &+& 2y &=& 3
    \end{array}
    \right.
\end{displaymath}
Jeśli do pierwszego równania dodamy drugie po obu stronach, uzyskamy od razu
równanie z jedną zmienną ($x$),
co sprawiło że jedną operacją zrobiliśmy 2 kroki z metody wyżej!
(żeby pokazać, co się zrobiło, można to odręcznie zapisać z boku jako ,,$| +$ drugie równanie'')
Można jednak powiedzieć ,,Ok, ale to działa tylko, gdy jeden z
współczynników jest taki sam (lub pomnożony przez -1) jak odpowiedni
współczynnik z innego równania!''.
Warto wtedy zauważyć, że przed dodaniem innego równania można je
pomnożyć przez jakąś liczbę. Już nawet wspomnieliśmy o przypadku, gdy przed dodaniem
mnożymy przez $-1$, wychodzi nam wtedy odejmowanie! Przykład:
\begin{displaymath}
  \left\{
    \begin{array}{lllll}
      2x &-& 7y &=& 5\\
      \tfrac12x &+& 5y &=& 3
    \end{array}
    \right.
\end{displaymath}
W powyższym równaniu możemy od pierwszego równania odjąć 4-krotność drugiego,
albo od drugiego równania $\tfrac14$ drugiego.

To pokazuje nam również drugą ciekawą rzecz: nawet jeśli akurat nie chcemy skorzystać z
tej metody, to możemy mnożyć (a także i dzielić!) wszystkie współczynniki równania tak,
by nam się łatwiej liczyło. Na przykład w powyższym układzie możemy pozbyć się ułamka
w drugim równaniu poprzez pomnożenie wszystkich współczynników przez 2.


Drugim przykładem jest poniższy układ:
\begin{displaymath}
  \left\{
    \begin{array}{lllll}
      4x &+& 8y &=& 16\\
      3x &-& 5y &=& 2
    \end{array}
    \right.
\end{displaymath}
Trzeba przyznać, że to pierwsze równanie aż woła o podzielenie obu stron przez 4!
Zademonstruję pełne rozwiązanie tego układu:
\begin{align*}
  &\left\{
    \begin{array}{llllll}
      4x &+& 8y &=& 16&|/4 \text{ (''/'' to jest oczywiście znak dzielenia)}\\
      3x &-& 5y &=& 2 &
    \end{array}
  \right.\\
  &\left\{
    \begin{array}{llllll}
      x &+& 2y &=& 4&\\
      3x &-& 5y &=& 2 &|-3\cdot  \text{pierwsze równanie}
    \end{array}
  \right.\\
  &\left\{
    \begin{array}{llllll}
      x &+& 2y &=& 4&\\
       &-& 11y &=& -10 &|\cdot(-1)
    \end{array}
  \right.\\
  &\left\{
    \begin{array}{llllll}
      x &+& 2y &=& 4&\\
       && 11y &=& 10 &|/11
    \end{array}
  \right.\\
  &\left\{
    \begin{array}{llllll}
      x &+& 2y &=& 4&|\text{ wstawiamy $y$ z 2 równania}\\
       && y &=& \frac{10}{11}&
    \end{array}
  \right.\\
  &\left\{
    \begin{array}{llllll}
      x &+& \frac{20}{11} &=& 4&|-\frac{20}{11}\\
       && y &=& \frac{10}{11}&
    \end{array}
  \right.\\
  &\left\{
    \begin{array}{llllll}
      x && &=& \frac{35}{11}&\\[5pt]
       && y &=& \frac{10}{11}&
    \end{array}
  \right.\text{I gotowe!}\\
\end{align*}


\textit{Mogłeś zauważyć, że zapisywałem te równania tak, by zawsze wyrażenia z tą samą zmienną
były ciurkiem pod sobą (szczególnie to widać w ostatnim układzie). O ile rozwiązując
zadanie na kartce robienie tego jest nadgorliwością, o tyle tutaj jest to przydatną
ilustracją. Nie tylko pokazuje jasno co się zmienia z równania na równanie, ale jeszcze
zwraca uwagę na to, że równanie bez $y$ to to samo, co równanie z $y$ pomnożonym
przez zero. Na razie nie jest to ważne, ale warto trzymać to sobie w głowie, bo
przyda się do zrozumienia trzeciego sposobu rozwiązywania układów równań...}

\subsection{Zastosowanie}
Widzieliśmy już zastosowanie układów równań liniowych do kilku rzeczy, przypomnijmy je.

\subsubsection{Interpretacja geometryczna}
Po pierwsze, równanie liniowe, po przekształceniu do formy $y=ax+b$ tworzy
nam \textbf{równanie} (linii...) \textbf{prostej}. $a$ to \textbf{współczynnik kierunkowy}, czyli
liczba mówiąca nam jak bardzo nachylona jest prosta. Gdy $a$ jest dodatnie, funkcja jest
rosnąca (czyli linia idzie do góry gdy $x$ idzie w prawo). Gdy $a$ jest ujemne,
funkcja jest malejąca (nigdy nie zgadniesz, jak wpływa to na linię). \textit{Możesz
za to spróbować zgadnąć co się dzieje gdy $a=0$...} Z kolei $b$ to \textbf{wyraz wolny}.
Nazwa sugeruje pewien rodzaj swobody - tutaj odnosi się to do tego, że zmiana $b$ na
większe lub mniejsze powoduje przesunięcie całej linii odpowiednio do góry lub w dół.
Równanie prostej jest równaniem prostej dlatego, że każdy punkt $(x, y)$ na tej prostej
musi je spełniać.

Zatem co dostaniemy, gdy mamy układ dwóch równań liniowych? Otóż mamy wtedy po prostu dwie
linie. Jednak jak wcześniej zauważyliśmy, w układzie równań chodzi o to, by oba te
równania były spełnione \textit{naraz}. Co to nam daje w kwestii punktów? Jak łatwo
zauważyć, otrzymamy tylko te punkty, które spełniają oba równania. A więc na obrazku
będą to punkty wspólne tych prostych.
\subitem Gdy układ jest oznaczony, proste się przecinają w jednym punkcie.
\subitem Gdy układ jest nieoznaczony, proste się pokrywają.
\subitem Gdy układ jest sprzeczny, proste są równoległe.


Widzimy zatem, że proste mają bardzo głębokie powiązania z równaniami liniowymi.
Spójrzmy sobie na kolejne z takich powiązań.

\subsubsection{Prosta przechodząca przez dwa punkty}

A co jeśli mamy dwa punkty $(x, y)$, i chcemy wyznaczyć równanie prostej, która przez nie przechodzi?
Widać, że będzie tylko jedna taka prosta. Będzie więc miała równanie: $y=ax+b$.
Żeby wyznaczyć równanie, potrzebujemy poznać $a$ oraz $b$. Skoro każdy punkt prostej
spełnia to równanie, to nasze dwa punkty również je spełniają. Możemy więc wstawić
oba punkty do równania i uzyskać układ równań nad dwiema zmiennymi: $a, b$.
Dla przykładu, niech punktami będą $(3, 7)$ oraz $(-2, 8)$.
Stwórzmy układ równań:
\begin{displaymath}
  \left\{
    \begin{array}{lllll}
      y &=& ax &+& b\\
      y &=& ax &+& b
    \end{array}
  \right.
\end{displaymath}
W pierwszym równaniu podstawiamy pod $(x, y)$ jeden punkt, a w drugim drugi.
\begin{displaymath}
  \left\{
    \begin{array}{lllll}
      7 &=& 3a &+& b\\
      8 &=& -2a &+& b
    \end{array}
  \right.
\end{displaymath}
Teraz mamy układ równań, który umiemy już rozwiązać! Dla pełności wyznaczmy równanie,
najpierw odejmując stronami drugie równanie od pierwszego:
\begin{align*}
  &\left\{
    \begin{array}{lllll}
      -1 &=& 5a && \text{(więc $a=-\frac15$)}\\
      8 &=& -2a &+& b
    \end{array}
  \right.\\
  &\left\{
    \begin{array}{lllll}
      a &=& -\frac15 &&\\[5px]
      8 &=& -\frac25 &+& b
    \end{array}
  \right.\\
  b&=8 + \frac25=\frac{42}5\\
  &\left\{
    \begin{array}{lllll}
      a &=& -\frac15 &&\\[5px]
      b &=& \frac{42}5 &&
    \end{array}
  \right.
\end{align*}
Otrzymujemy równanie prostej $y=-\frac15a + \frac{42}5$.
\subsubsection{Kiedy układy nie mają unikalnego rozwiązania?}
Układ jest nieoznaczony, gdy jedno równanie da się sprowadzić poprzez
\textit{elementarne działania} do drugiego.
Elementarne działanie to takie, w którym robimy działanie na obu stronach z użyciem
albo liczby, albo innego równania z układu. Chodzi po prostu o to, by
nie można było odjąć obu równań stronami od samych siebie i bohatersko stwierdzić, że
voilà, oba równania sprowadziliśmy do $0=0$. Można też zauważyć, że zawsze, gdy da się
sprowadzić elementarnymi działaniami równanie do drugiego, to można też je sprowadzić do
równania $0=0$ i vice versa
(po prostu odejmujemy (lub odpowiednio dodajemy) drugie z równań). Podobny argument
pokazuje nam, że można też sprowadzić do dowolnej równości, w której nie ma zmiennych
(pod warunkiem że jest prawdziwa!).
W poniższym przykładzie wystarczy podzielić drugie równanie przez $-3$:
\begin{displaymath}
  \left\{
    \begin{array}{lllll}
      4x &-& 8y &=& 15\\
      -12x &+& 24y &=& -45
    \end{array}
  \right.
\end{displaymath}


Układ jest sprzeczny, gdy takimi właśnie elementarnymi działaniami możemy sprowadzić go
do fałszu (sprzeczności), czyli np. równania $0=1$.
W praktyce jest to bardzo podobny warunek do poprzedniego, bo oba najłatwiej jest
sprawdzić poprzez sprowadzenie obu równań do postaci standardowej, ale w taki sposób,
aby przy obu zmiennych były te same współczynniki. Jeśli po drugiej stronie równań
wyjdą nam te same liczby, układ jest nieoznaczony. Jeśli wyjdą różne, układ jest sprzeczny.

\end{document}
