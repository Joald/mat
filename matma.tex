\documentclass{article}
\usepackage[english,polish]{babel}
\usepackage{polski}
\usepackage[utf8]{inputenc}
\usepackage{amsfonts}
\usepackage{soul}
\usepackage{textcomp}
\usepackage{color}
\usepackage{enumitem}
\usepackage{amsmath}
%\usepackage{amssymb}
\usepackage{mathtools}
\usepackage{listings}
\usepackage{amsthm}
\usepackage[a4paper, total={6in, 10in}]{geometry}
\definecolor{dkgreen}{rgb}{0,0.6,0}
\definecolor{gray}{rgb}{0.5,0.5,0.5}
\definecolor{mauve}{rgb}{0.58,0,0.82}
\lstset{frame=tb,
    language=Bash,
    aboveskip=3mm,
    belowskip=3mm,
    showstringspaces=false,
    columns=flexible,
    basicstyle={\small\ttfamily},
    numbers=none,
    numberstyle=\tiny\color{gray},
    keywordstyle=\color{blue},
    commentstyle=\color{dkgreen},
    stringstyle=\color{mauve},
    breaklines=true,
    breakatwhitespace=true,
    escapeinside={(*}{*)},          % if you want to add LaTeX within your code
    tabsize=4
}
\theoremstyle{remark}
\newtheorem*{lemma}{Lemma}

\title{Warte umienia, ciekawe zagadnienia}
\author{}
\date{}
\begin{document}
\maketitle
\section{Wzory skróconego mnożenia}
\begin{align}
  (a+b)^2&=a^2+2ab+b^2\\
  (a-b)^2&=a^2-2ab+b^2\\
  a^2-b^2&=(a+b)(a-b)
\end{align}
\section{Układy równań liniowych}
\subsection{Terminologia i notacja}
Po to, by było do bólu jasne, o czym jest mowa później.
(Potencjalnie) nowe pojęcia wprowadzam pogrubione.
\begin{itemize}
  \item \textbf{Równaniem liniowym} nad zmiennymi $x, y$ 
  nazywamy każde równanie, które można sprowadzić do postaci
  $ax + by = c$, gdzie $a, b, c$ są znane, np. $3x+2y=-8$. 
  Analogicznie definiujemy równanie liniowe 
  nad większymi ilościami zmiennych 
  (ważne jest, by każda zmienna była pomnożona tylko przez
   stałą, tj. coś co nie jest zmienną).
  \item \textbf{Zbiorem} nazywamy kolekcję elementów (no shit, Sherlock), 
    która nie zawiera dwóch tych samych elementów.
  \begin{itemize}
    \item Zbiór pusty oznaczamy przez $\emptyset$.
    \item Zbiór zawierający pojedynczy element (w tym przypadku liczbę 1)
  oznaczamy przez $\{1\}$ (więcej elementów można dodać po przecinku, np. $\{1, 2, 3\}$
  ale nie będzie to tutaj potrzebne). Jednoelementowy zbiór nazywa się \textbf{singleton}.
    \item Zbiór wszystkich liczb rzeczywistych oznaczamy przez $\mathbb R$ 
  (odręcznie pisane jako wielka litera R z podwójną nóżką).
  \end{itemize}
  \item \textbf{Układem równań liniowych} nazywamy ciąg równań, które jednocześnie uznajemy za prawdziwe.
  Przykłady:
  \begin{align*}
    &\left\{ 
    \begin{array}{lllll}
      3x &+& 2y &=& 5\\
      4x &+& 7y &=& 3
    \end{array}
    \right. \\
    &\left\{ 
    \begin{array}{lllllll}
      \tfrac12x &+& 2y &-&z &=& 5\\
       && 23y &-&77z &=& 100\\
      4x &+& \tfrac74y &+&9z &=& 3
    \end{array}
    \right.
  \end{align*}
  Oczywiście, układ równań jest zdefiniowany nad tymi zmiennymi, nad 
  którymi są zdefiniowane jego równania. 
  Ale, \textit{możemy} też sobie powiedzieć, 
  że pewien konkretny układ jest nad innymi zmiennymi.
  Wtedy powiemy sobie, że każde z jego równań również
  jest nad tymi zmiennymi.
  \item Do każdego układu równań liniowych określamy \textbf{zbiór rozwiązań
  dla zmiennej} $x$. 
  Chodzi o wszystkie możliwe wartości, jakie ta zmienna może przyjmować 
  pod warunkiem, że wszystkie równania są prawdziwe.
  Są trzy przypadki:
  \begin{center}
    \begin{tabular}{|c|c|}
      \hline
      możliwe wartości $x$ &zbiór rozwiązań dla $x$\\
      \hline\hline
      tylko jedna wartość (oznaczę ją jako $w$) & $\{w\}$\\
      \hline
      $x$ jest dowolne & $\mathbb R$\\
      \hline
      żadna wartość nie jest możliwa & $\emptyset$\\
      \hline
    \end{tabular}
  \end{center}
  
  \item \textbf{Rozwiązaniem} układu równań liniowych nazywamy 
  przypisanie zmiennych, nad którym ten układ jest zdefiniowany do 
  zbiorów rozwiązań dla tych zmiennych. Banalny przykład:
  Rozwiązaniem układu równań $
    \left\{ 
      \begin{array}{l}
        x=3\\
        y=2
      \end{array}
    \right.$ jest przypisanie $x \to \{3\}, y \to \{2\}$.
  \textit{Można zauważyć, że takie przypisanie też jest funkcją.}
  
  \item Układ równań określamy jako \textbf{oznaczony}, jeśli jego
  rozwiązanie przypisuje do \textit{każdej} zmiennej singleton.
  Np. rozwiązanie z poprzedniego podpunktu jest dowodem na to, że ów układ
  jest oznaczony. Mówimy, że taki układ \textbf{ma jedno rozwiązanie}.
  \item Układ równań określamy jako \textbf{sprzeczny}, jeśli jego
  rozwiązanie przypisuje do \textit{przynajmniej jednej} zmiennej zbiór pusty ($\emptyset$).
  Mówimy, że taki układ \textbf{nie ma rozwiązań}.
  \item W przeciwnym wypadku, tj. jeśli rozwiązanie przypisuje do 
  \textit{przynajmniej jednej} ze zmiennych $\mathbb R$, 
  a do innych (jeśli istnieją) singleton,
  układ nazywamy \textbf{nieoznaczonym}.
\end{itemize}
\subsection{Spostrzeżenia przydatne do rozwiązywania zadań}
Jak już zauważyliśmy IRL, układy równań z dwiema zmiennymi można rozwiązywać w 
następujący sposób:
\begin{enumerate}
  \item Przekształć jedno z równań tak, by po jednej stronie była 
  jedna z zmiennych, np. $x$. W ten sposób uzyskujemy
  wzór na $x$ zależny tylko od $y$.
  \item Wstaw tak wyznaczony $x$ do drugiego równania, 
  w ten sposób uzyskując równanie bez $x$.
  \item Z tego równania można łatwo wyznaczyć wartość $y$.
  \item Znając $y$, wstawiamy tą wartość do wzoru wyznaczonego w punkcie 1 
  i otrzymujemy wartość $x$.
\end{enumerate}
Jak nietrudno zauważyć, ten sposób działania zadziała również przy większej 
liczbie zmiennych niż 2. Jednak szybko się okazuje, że takie liczenie jest powolne
i nudne, komu by się chciało w ten sposób rozwiązywać. Spójrzmy zatem na krok 1, czy nie
dałoby się go przyspieszyć. Zwróćmy uwagę, że przekształceń na równaniu dokonujemy 
poprzez wykonywanie działań po obu stronach równania. Np. 
dodając/odejmując coś do/od obu stron lub mnożąc/dzieląc obie strony przez jakąś liczbę
(oczywiście pamiętając, żeby nie dzielić przez 0!). 
Zauważmy jednak, że przekształcając jedno z równań, mamy pewną wiedzę, 
którą możemy wykorzystać - resztę równań! W końcu one też są prawdziwe, więc możemy z
jednej strony naszego równania dodać 
(odjąć, nawet pomnożyć i podzielić, choć to jest radziej przydatne)
jedną stronę innego równania, a z drugiej drugą.
Ale po co nam to? Przydatność tego sposobu łatwo zilustrować takim przykładem:
\begin{displaymath}
  \left\{ 
    \begin{array}{lllll}
      3x &-& 2y &=& 5\\
      4x &+& 2y &=& 3
    \end{array}
    \right.
\end{displaymath}
Jeśli do pierwszego równania dodamy drugie po obu stronach, uzyskamy od razu
równanie z jedną zmienną ($x$), 
co sprawiło że jedną operacją zrobiliśmy 2 kroki z metody wyżej!
(żeby pokazać, co się zrobiło, można to odręcznie zapisać z boku jako ,,$| +$ drugie równanie'')
Można jednak powiedzieć ,,Ok, ale to działa tylko, gdy jeden z 
współczynników jest taki sam (lub pomnożony przez -1) jak odpowiedni 
współczynnik z innego równania!''.
Warto wtedy zauważyć, że przed dodaniem innego równania można je 
pomnożyć przez jakąś liczbę. Już nawet wspomnieliśmy o przypadku, gdy przed dodaniem
mnożymy przez $-1$, wychodzi nam wtedy odejmowanie! Przykład:
\begin{displaymath}
  \left\{ 
    \begin{array}{lllll}
      2x &-& 7y &=& 5\\
      \tfrac12x &+& 5y &=& 3
    \end{array}
    \right.
\end{displaymath}
W powyższym równaniu możemy od pierwszego równania odjąć 4-krotność drugiego, 
albo od drugiego równania $\tfrac14$ drugiego.

\end{document}
